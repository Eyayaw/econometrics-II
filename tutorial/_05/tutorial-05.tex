% Options for packages loaded elsewhere
\PassOptionsToPackage{unicode}{hyperref}
\PassOptionsToPackage{hyphens}{url}
\PassOptionsToPackage{dvipsnames,svgnames*,x11names*}{xcolor}
%
\documentclass[
]{article}
\usepackage{lmodern}
\usepackage{amssymb,amsmath}
\usepackage{ifxetex,ifluatex}
\ifnum 0\ifxetex 1\fi\ifluatex 1\fi=0 % if pdftex
  \usepackage[T1]{fontenc}
  \usepackage[utf8]{inputenc}
  \usepackage{textcomp} % provide euro and other symbols
\else % if luatex or xetex
  \usepackage{unicode-math}
  \defaultfontfeatures{Scale=MatchLowercase}
  \defaultfontfeatures[\rmfamily]{Ligatures=TeX,Scale=1}
\fi
% Use upquote if available, for straight quotes in verbatim environments
\IfFileExists{upquote.sty}{\usepackage{upquote}}{}
\IfFileExists{microtype.sty}{% use microtype if available
  \usepackage[]{microtype}
  \UseMicrotypeSet[protrusion]{basicmath} % disable protrusion for tt fonts
}{}
\makeatletter
\@ifundefined{KOMAClassName}{% if non-KOMA class
  \IfFileExists{parskip.sty}{%
    \usepackage{parskip}
  }{% else
    \setlength{\parindent}{0pt}
    \setlength{\parskip}{6pt plus 2pt minus 1pt}}
}{% if KOMA class
  \KOMAoptions{parskip=half}}
\makeatother
\usepackage{xcolor}
\IfFileExists{xurl.sty}{\usepackage{xurl}}{} % add URL line breaks if available
\IfFileExists{bookmark.sty}{\usepackage{bookmark}}{\usepackage{hyperref}}
\hypersetup{
  pdftitle={IV Estimation: Strangetown},
  pdfauthor={Eyayaw Beze},
  colorlinks=true,
  linkcolor=Maroon,
  filecolor=Maroon,
  citecolor=Blue,
  urlcolor=Blue,
  pdfcreator={LaTeX via pandoc}}
\urlstyle{same} % disable monospaced font for URLs
\usepackage[margin=1in]{geometry}
\usepackage{color}
\usepackage{fancyvrb}
\newcommand{\VerbBar}{|}
\newcommand{\VERB}{\Verb[commandchars=\\\{\}]}
\DefineVerbatimEnvironment{Highlighting}{Verbatim}{commandchars=\\\{\}}
% Add ',fontsize=\small' for more characters per line
\usepackage{framed}
\definecolor{shadecolor}{RGB}{248,248,248}
\newenvironment{Shaded}{\begin{snugshade}}{\end{snugshade}}
\newcommand{\AlertTok}[1]{\textcolor[rgb]{0.94,0.16,0.16}{#1}}
\newcommand{\AnnotationTok}[1]{\textcolor[rgb]{0.56,0.35,0.01}{\textbf{\textit{#1}}}}
\newcommand{\AttributeTok}[1]{\textcolor[rgb]{0.77,0.63,0.00}{#1}}
\newcommand{\BaseNTok}[1]{\textcolor[rgb]{0.00,0.00,0.81}{#1}}
\newcommand{\BuiltInTok}[1]{#1}
\newcommand{\CharTok}[1]{\textcolor[rgb]{0.31,0.60,0.02}{#1}}
\newcommand{\CommentTok}[1]{\textcolor[rgb]{0.56,0.35,0.01}{\textit{#1}}}
\newcommand{\CommentVarTok}[1]{\textcolor[rgb]{0.56,0.35,0.01}{\textbf{\textit{#1}}}}
\newcommand{\ConstantTok}[1]{\textcolor[rgb]{0.00,0.00,0.00}{#1}}
\newcommand{\ControlFlowTok}[1]{\textcolor[rgb]{0.13,0.29,0.53}{\textbf{#1}}}
\newcommand{\DataTypeTok}[1]{\textcolor[rgb]{0.13,0.29,0.53}{#1}}
\newcommand{\DecValTok}[1]{\textcolor[rgb]{0.00,0.00,0.81}{#1}}
\newcommand{\DocumentationTok}[1]{\textcolor[rgb]{0.56,0.35,0.01}{\textbf{\textit{#1}}}}
\newcommand{\ErrorTok}[1]{\textcolor[rgb]{0.64,0.00,0.00}{\textbf{#1}}}
\newcommand{\ExtensionTok}[1]{#1}
\newcommand{\FloatTok}[1]{\textcolor[rgb]{0.00,0.00,0.81}{#1}}
\newcommand{\FunctionTok}[1]{\textcolor[rgb]{0.00,0.00,0.00}{#1}}
\newcommand{\ImportTok}[1]{#1}
\newcommand{\InformationTok}[1]{\textcolor[rgb]{0.56,0.35,0.01}{\textbf{\textit{#1}}}}
\newcommand{\KeywordTok}[1]{\textcolor[rgb]{0.13,0.29,0.53}{\textbf{#1}}}
\newcommand{\NormalTok}[1]{#1}
\newcommand{\OperatorTok}[1]{\textcolor[rgb]{0.81,0.36,0.00}{\textbf{#1}}}
\newcommand{\OtherTok}[1]{\textcolor[rgb]{0.56,0.35,0.01}{#1}}
\newcommand{\PreprocessorTok}[1]{\textcolor[rgb]{0.56,0.35,0.01}{\textit{#1}}}
\newcommand{\RegionMarkerTok}[1]{#1}
\newcommand{\SpecialCharTok}[1]{\textcolor[rgb]{0.00,0.00,0.00}{#1}}
\newcommand{\SpecialStringTok}[1]{\textcolor[rgb]{0.31,0.60,0.02}{#1}}
\newcommand{\StringTok}[1]{\textcolor[rgb]{0.31,0.60,0.02}{#1}}
\newcommand{\VariableTok}[1]{\textcolor[rgb]{0.00,0.00,0.00}{#1}}
\newcommand{\VerbatimStringTok}[1]{\textcolor[rgb]{0.31,0.60,0.02}{#1}}
\newcommand{\WarningTok}[1]{\textcolor[rgb]{0.56,0.35,0.01}{\textbf{\textit{#1}}}}
\usepackage{graphicx}
\makeatletter
\def\maxwidth{\ifdim\Gin@nat@width>\linewidth\linewidth\else\Gin@nat@width\fi}
\def\maxheight{\ifdim\Gin@nat@height>\textheight\textheight\else\Gin@nat@height\fi}
\makeatother
% Scale images if necessary, so that they will not overflow the page
% margins by default, and it is still possible to overwrite the defaults
% using explicit options in \includegraphics[width, height, ...]{}
\setkeys{Gin}{width=\maxwidth,height=\maxheight,keepaspectratio}
% Set default figure placement to htbp
\makeatletter
\def\fps@figure{htbp}
\makeatother
\setlength{\emergencystretch}{3em} % prevent overfull lines
\providecommand{\tightlist}{%
  \setlength{\itemsep}{0pt}\setlength{\parskip}{0pt}}
\setcounter{secnumdepth}{-\maxdimen} % remove section numbering

\title{IV Estimation: Strangetown}
\author{Eyayaw Beze}
\date{July 11, 2020}

\begin{document}
\maketitle

Let \(y_i\) be a measure of good health, and \(D_i\) an indicator for
smoking \[ D_i = \left\{
 \begin{array}{lr}
 1, \quad if \;i \; smokes\\
 0, \quad otherwise
 \end{array}
 \right.
 \]

\[ Z_i = \left\{
 \begin{array}{lr}
 1, \quad if \;i \; \text{received a pack of cigarettes}\\
 0, \quad otherwise
 \end{array}
 \right.
 \]

The true model reads: \begin{equation}
 y_{i}=\alpha+\beta_{1}^{\text {true}} D_{i}+\beta_{2} w_{i}+\epsilon_{i}\label{eq:1}
 \end{equation} with \(w_{i}\) denoting individual income. But since you
do not observe income you can only estimate the following model:
\begin{equation}
 y_{i}=\alpha+\beta_{1} D_{i}+u_{i}\label{eq:2}
 \end{equation} a) Over- or under-estimate \(\beta_{1}^{\text {true}}\)
if we don't observe wage in (\ref{eq:1})? Show the inconsistency of the
coefficient if we estimate (\ref{eq:2}) instead?

\begin{Shaded}
\begin{Highlighting}[]
\CommentTok{\# data generating process}
\NormalTok{dgp \textless{}{-}}\StringTok{ }\ControlFlowTok{function}\NormalTok{(n, y.bar, d, z) \{}
\NormalTok{  sigma \textless{}{-}}\StringTok{ }\DecValTok{1} \OperatorTok{/}\StringTok{ }\KeywordTok{sqrt}\NormalTok{(n)}
\NormalTok{  y \textless{}{-}}\StringTok{ }\NormalTok{y.bar }\OperatorTok{+}\StringTok{ }\NormalTok{sigma }\OperatorTok{*}\StringTok{ }\KeywordTok{scale}\NormalTok{(}\KeywordTok{rnorm}\NormalTok{(n)) }\CommentTok{\# with fixed mean and sd; }
                                       \CommentTok{\# mean(y) == y.bar, sd(x) == sigma}
  \KeywordTok{return}\NormalTok{(}\KeywordTok{data.frame}\NormalTok{(y, }\DataTypeTok{D =} \KeywordTok{rep}\NormalTok{(d, n), }\DataTypeTok{Z =} \KeywordTok{rep}\NormalTok{(z, n)))}
\NormalTok{\}}

\CommentTok{\# to vectorize over n, y.bar, D, and Z}
\NormalTok{map\_df \textless{}{-}}\StringTok{ }\ControlFlowTok{function}\NormalTok{(..., f, }\DataTypeTok{binder =}\NormalTok{ rbind) \{}
  \KeywordTok{return}\NormalTok{(}\KeywordTok{as.data.frame}\NormalTok{(}\KeywordTok{do.call}\NormalTok{(binder, }\KeywordTok{Map}\NormalTok{(f, ...))))}
\NormalTok{\}}
\end{Highlighting}
\end{Shaded}

\begin{Shaded}
\begin{Highlighting}[]
\NormalTok{n \textless{}{-}}\StringTok{ }\DecValTok{70} \CommentTok{\#\# sample size}
\CommentTok{\# Di = Zi = 0}
\NormalTok{n00 \textless{}{-}}\StringTok{ }\DecValTok{30}
\NormalTok{y00.bar \textless{}{-}}\StringTok{ }\FloatTok{1.0}

\CommentTok{\# Di = 0; Zi = 1}
\NormalTok{n01 \textless{}{-}}\StringTok{ }\DecValTok{10}
\NormalTok{y01.bar \textless{}{-}}\StringTok{ }\FloatTok{0.8}
\CommentTok{\# Di = 1; Zi = 0}
\NormalTok{n10 \textless{}{-}}\StringTok{ }\DecValTok{20}
\NormalTok{y10.bar \textless{}{-}}\StringTok{ }\FloatTok{1.5}
\CommentTok{\# Di = Zi = 1}
\NormalTok{n11 \textless{}{-}}\StringTok{ }\DecValTok{10}
\NormalTok{y11.bar \textless{}{-}}\StringTok{ }\FloatTok{1.2}

\NormalTok{n.vec \textless{}{-}}\StringTok{ }\KeywordTok{c}\NormalTok{(n00, n01, n10, n11)}
\NormalTok{y.bar \textless{}{-}}\StringTok{ }\KeywordTok{c}\NormalTok{(y00.bar, y01.bar, y10.bar, y11.bar)}
\NormalTok{D \textless{}{-}}\StringTok{ }\KeywordTok{c}\NormalTok{(}\DecValTok{0}\NormalTok{, }\DecValTok{0}\NormalTok{, }\DecValTok{1}\NormalTok{, }\DecValTok{1}\NormalTok{)}
\NormalTok{Z \textless{}{-}}\StringTok{ }\KeywordTok{c}\NormalTok{(}\DecValTok{0}\NormalTok{, }\DecValTok{1}\NormalTok{, }\DecValTok{0}\NormalTok{, }\DecValTok{1}\NormalTok{)}

\KeywordTok{set.seed}\NormalTok{(}\DecValTok{123}\NormalTok{) }\CommentTok{\# for reproducibility}
\NormalTok{toy\_data \textless{}{-}}\StringTok{ }\KeywordTok{map\_df}\NormalTok{(n.vec, y.bar, D, Z, }\DataTypeTok{f =}\NormalTok{ dgp)}

\KeywordTok{head}\NormalTok{(toy\_data) }\CommentTok{\# view 6 rows of the data}
\end{Highlighting}
\end{Shaded}

\begin{verbatim}
#>         y D Z
#> 1 0.90446 0 0
#> 2 0.96593 0 0
#> 3 1.29885 0 0
#> 4 1.02189 0 0
#> 5 1.03283 0 0
#> 6 1.32795 0 0
\end{verbatim}

\begin{Shaded}
\begin{Highlighting}[]
\NormalTok{subsets \textless{}{-}}\StringTok{ }\KeywordTok{alist}\NormalTok{(}
\NormalTok{  D }\OperatorTok{==}\StringTok{ }\DecValTok{0} \OperatorTok{\&}\StringTok{ }\NormalTok{Z }\OperatorTok{==}\StringTok{ }\DecValTok{0}\NormalTok{,}
\NormalTok{  D }\OperatorTok{==}\StringTok{ }\DecValTok{0} \OperatorTok{\&}\StringTok{ }\NormalTok{Z }\OperatorTok{==}\StringTok{ }\DecValTok{1}\NormalTok{,}
\NormalTok{  D }\OperatorTok{==}\StringTok{ }\DecValTok{1} \OperatorTok{\&}\StringTok{ }\NormalTok{Z }\OperatorTok{==}\StringTok{ }\DecValTok{0}\NormalTok{,}
\NormalTok{  D }\OperatorTok{==}\StringTok{ }\DecValTok{1} \OperatorTok{\&}\StringTok{ }\NormalTok{Z }\OperatorTok{==}\StringTok{ }\DecValTok{1}
\NormalTok{)}

\CommentTok{\# check whether the mean of the generated data matches the sample means}
\KeywordTok{sapply}\NormalTok{(subsets, }\ControlFlowTok{function}\NormalTok{(x) }\KeywordTok{mean}\NormalTok{(toy\_data[}\KeywordTok{eval}\NormalTok{(x, toy\_data), }\StringTok{"y"}\NormalTok{]))}
\end{Highlighting}
\end{Shaded}

\begin{verbatim}
#> [1] 1.0 0.8 1.5 1.2
\end{verbatim}

\begin{Shaded}
\begin{Highlighting}[]
\NormalTok{y.bar}
\end{Highlighting}
\end{Shaded}

\begin{verbatim}
#> [1] 1.0 0.8 1.5 1.2
\end{verbatim}

\begin{enumerate}
\def\labelenumi{\alph{enumi})}
\setcounter{enumi}{2}
\tightlist
\item
  Calculate \(\beta^{OLS}_1\) that you obtain by estimating equation (2)
  by OLS. Interpret the coefficient. \begin{gather}
  \begin{aligned}
  \beta^{OLS}_1 &= \frac{Cov(y_i, D_i)}{Var(D_i)} = \frac{E\left[y_i D_i\right]-E\left[y_i\right] E\left[D_i\right]}{E[D_i^2]-(E[D_i])^2} = \frac{E\left[y_i\mid D_i=1\right]-E\left[y_i\right] E[D_i]}{E[D_i]-(E[D_i])^2} \\
  & = \frac{\frac{(n_{10}\bar{y}_{10} + n_{11}\bar{y}_{11})}{n} - \frac{(n_{00}\bar{y}_{00} + n_{01}\bar{y}_{01} + n_{10}\bar{y}_{10} + n_{11}\bar{y}_{11})}{n}\frac{(n_{10} + n_{11})}{n}}{(\frac{n_{10} + n_{11}}{n}) - \left(\frac{n_{10} + n_{11}}{n}\right)^2}  \\
  &= \frac{(1/70)\left(20\cdot1.5 + 10\cdot1.2)\right) - \left[(1/70) \left(30\cdot1.0 + 10\cdot0.8 + 20\cdot1.5 + 10\cdot1.2\right)\right]\left[(1/70)(20 + 10)\right]}{(\frac{20 + 10}{70})-(\frac{20 + 10}{70})^2} \label{eq:3}
  \end{aligned}
  \end{gather} Note: \(E[D_i] = p\) and \(Var(D_i) = p(1-p)\) where p is
  the probability that \(D_i\) takes on 1---\textbf{since
  \(\boldsymbol{D_i}\) is a Bernoulli random variable}. In our case,
  \(Pr(D_i=1) = p = \frac{n_{10} + n_{11}}{n_{00} + n_{01} + n_{10} + n_{11}} = (20 + 10) / 70 = 3/7\).
\end{enumerate}

\hypertarget{iv-estimation-latewald}{%
\section{IV Estimation: LATE/Wald}\label{iv-estimation-latewald}}

By (\ref{eq:3}),

\begin{Shaded}
\begin{Highlighting}[]
\NormalTok{b\_ols \textless{}{-}}\StringTok{ }\NormalTok{((}\DecValTok{1} \OperatorTok{/}\StringTok{ }\NormalTok{n) }\OperatorTok{*}\StringTok{ }\NormalTok{(n10 }\OperatorTok{*}\StringTok{ }\NormalTok{y10.bar }\OperatorTok{+}\StringTok{ }\NormalTok{n11 }\OperatorTok{*}\StringTok{ }\NormalTok{y11.bar) }\OperatorTok{{-}}
\StringTok{  }\NormalTok{((}\DecValTok{1} \OperatorTok{/}\StringTok{ }\NormalTok{n) }\OperatorTok{*}\StringTok{ }\NormalTok{(n00 }\OperatorTok{*}\StringTok{ }\NormalTok{y00.bar }\OperatorTok{+}\StringTok{ }\NormalTok{n01 }\OperatorTok{*}\StringTok{ }\NormalTok{y01.bar }\OperatorTok{+}\StringTok{ }\NormalTok{n10 }\OperatorTok{*}\StringTok{ }\NormalTok{y10.bar }\OperatorTok{+}\StringTok{ }\NormalTok{n11 }\OperatorTok{*}\StringTok{ }\NormalTok{y11.bar) }\OperatorTok{*}\StringTok{ }\NormalTok{(n10 }\OperatorTok{+}\StringTok{ }\NormalTok{n11) }\OperatorTok{/}\StringTok{ }\NormalTok{n}
\NormalTok{  )) }\OperatorTok{/}
\StringTok{  }\NormalTok{((n10 }\OperatorTok{+}\StringTok{ }\NormalTok{n11) }\OperatorTok{/}\StringTok{ }\NormalTok{n }\OperatorTok{{-}}\StringTok{ }\NormalTok{((n10 }\OperatorTok{+}\StringTok{ }\NormalTok{n11) }\OperatorTok{/}\StringTok{ }\NormalTok{n)}\OperatorTok{\^{}}\DecValTok{2}\NormalTok{)}
\end{Highlighting}
\end{Shaded}

\(\hat{\beta}^{OLS}_1=\) 0.45.

Or using the \texttt{covariance} and \texttt{variance} formula:

\begin{Shaded}
\begin{Highlighting}[]
\NormalTok{beta\_ols \textless{}{-}}\StringTok{ }\KeywordTok{with}\NormalTok{(toy\_data, }\KeywordTok{cov}\NormalTok{(y, D) }\OperatorTok{/}\StringTok{ }\KeywordTok{var}\NormalTok{(D))}
\NormalTok{alpha \textless{}{-}}\StringTok{ }\KeywordTok{with}\NormalTok{(toy\_data, }\KeywordTok{mean}\NormalTok{(y) }\OperatorTok{{-}}\StringTok{ }\NormalTok{beta\_ols }\OperatorTok{*}\StringTok{ }\KeywordTok{mean}\NormalTok{(D))}
\end{Highlighting}
\end{Shaded}

\(\hat{\beta}^{OLS}_1=\) 0.45 and \(\hat\alpha\) = 0.95.

Or using \texttt{lm}---a linear model estimation workhorse in
\texttt{R}:

\begin{Shaded}
\begin{Highlighting}[]
\KeywordTok{lm}\NormalTok{(y }\OperatorTok{\textasciitilde{}}\StringTok{ }\NormalTok{D, toy\_data)}\OperatorTok{$}\NormalTok{coefficients}
\end{Highlighting}
\end{Shaded}

\begin{verbatim}
#> (Intercept)           D 
#>        0.95        0.45
\end{verbatim}

\begin{enumerate}
\def\labelenumi{(\alph{enumi})}
\setcounter{enumi}{3}
\tightlist
\item
  Calculate \({\beta}^{IV}_1\) and discuss your result w.r.t. to the
  previous findings. Using The Wald Estimator: \[
  {\beta}^{IV}_1=\frac{\mathbb{E}\left(Y_{i} \mid z_{i}=1\right)-\mathbb{E}\left(Y_{i} \mid z_{i}=0\right)}{\mathbb{E}\left(D_{i} \mid z_{i}=1\right)-\mathbb{E}\left(D_{i} \mid z_{i}=0\right)}
  \]
\end{enumerate}

\begin{Shaded}
\begin{Highlighting}[]
\CommentTok{\# Wald Estimator }
\KeywordTok{with}\NormalTok{(}
\NormalTok{  toy\_data,}
\NormalTok{  (}\KeywordTok{mean}\NormalTok{(y[Z }\OperatorTok{==}\StringTok{ }\DecValTok{1}\NormalTok{]) }\OperatorTok{{-}}\StringTok{ }\KeywordTok{mean}\NormalTok{(y[Z }\OperatorTok{==}\StringTok{ }\DecValTok{0}\NormalTok{])) }\OperatorTok{/}\StringTok{ }\NormalTok{(}\KeywordTok{mean}\NormalTok{(D[Z }\OperatorTok{==}\StringTok{ }\DecValTok{1}\NormalTok{]) }\OperatorTok{{-}}\StringTok{ }\KeywordTok{mean}\NormalTok{(D[Z }\OperatorTok{==}\StringTok{ }\DecValTok{0}\NormalTok{]))}
\NormalTok{)}
\end{Highlighting}
\end{Shaded}

\begin{verbatim}
#> [1] -2
\end{verbatim}

Using matrix notation:

\[
\widehat{\boldsymbol{\beta}}_{I V}=\left[\mathbf{Z^\prime} \mathbf{X}\right]^{-1} [\mathbf{Z^\prime} y]
\]

\begin{Shaded}
\begin{Highlighting}[]
\CommentTok{\# the [1, 1, 1,...]\textquotesingle{} is instrumented with itself}
\KeywordTok{with}\NormalTok{(toy\_data, \{}
\NormalTok{  constant \textless{}{-}}\StringTok{ }\KeywordTok{rep}\NormalTok{(}\DecValTok{1}\NormalTok{, }\KeywordTok{length}\NormalTok{(D))}
\NormalTok{  Z.mat \textless{}{-}}\StringTok{ }\KeywordTok{cbind}\NormalTok{(constant, Z) }\CommentTok{\# Z.mat = [[1, 1,..., 1]\textquotesingle{}, Z]}
\NormalTok{  D.mat \textless{}{-}}\StringTok{ }\KeywordTok{cbind}\NormalTok{(constant, D) }\CommentTok{\# [[1, 1,..., 1]\textquotesingle{}, D]}
  \KeywordTok{solve}\NormalTok{(}\KeywordTok{t}\NormalTok{(Z.mat) }\OperatorTok{\%*\%}\StringTok{ }\NormalTok{D.mat) }\OperatorTok{\%*\%}\StringTok{ }\KeywordTok{t}\NormalTok{(Z.mat) }\OperatorTok{\%*\%}\StringTok{ }\NormalTok{y}
\NormalTok{\})}
\end{Highlighting}
\end{Shaded}

\begin{verbatim}
#>          [,1]
#> constant    2
#> D          -2
\end{verbatim}

The Wald estimand of \(\widehat{\boldsymbol{\beta}}_{I V}\) can be
interpreted as the effect of smoking on those whose treatment status can
be changed by the instrument. The effect of smoking on health of those
who smoked because they were given packs of cigarettes, but would not
otherwise have smoked. This obviously excludes voluntary smokers and
those who did not, but it includes smokers for whom receiving the
cigarettes was important.

\begin{Shaded}
\begin{Highlighting}[]
\CommentTok{\# Or using AER package}
\KeywordTok{library}\NormalTok{(AER)}
\KeywordTok{library}\NormalTok{(stargazer)}
\NormalTok{model\_iv \textless{}{-}}\StringTok{ }\KeywordTok{ivreg}\NormalTok{(y }\OperatorTok{\textasciitilde{}}\StringTok{ }\NormalTok{D }\OperatorTok{|}\StringTok{ }\NormalTok{Z, }\DataTypeTok{data =}\NormalTok{ toy\_data)}
\KeywordTok{stargazer}\NormalTok{(model\_iv, }\DataTypeTok{header =} \OtherTok{FALSE}\NormalTok{)}
\end{Highlighting}
\end{Shaded}

\begin{table}[!htbp] \centering 
  \caption{} 
  \label{} 
\begin{tabular}{@{\extracolsep{5pt}}lc} 
\\[-1.8ex]\hline 
\hline \\[-1.8ex] 
 & \multicolumn{1}{c}{\textit{Dependent variable:}} \\ 
\cline{2-2} 
\\[-1.8ex] & y \\ 
\hline \\[-1.8ex] 
 D & $-$2.000 \\ 
  & (3.327) \\ 
  & \\ 
 Constant & 2.000 \\ 
  & (1.434) \\ 
  & \\ 
\hline \\[-1.8ex] 
Observations & 70 \\ 
R$^{2}$ & $-$12.293 \\ 
Adjusted R$^{2}$ & $-$12.489 \\ 
Residual Std. Error & 1.257 (df = 68) \\ 
\hline 
\hline \\[-1.8ex] 
\textit{Note:}  & \multicolumn{1}{r}{$^{*}$p$<$0.1; $^{**}$p$<$0.05; $^{***}$p$<$0.01} \\ 
\end{tabular} 
\end{table}

\begin{itemize}
\tightlist
\item
  \textbf{Compliers}. The subpopulation with \(d_{1i} = 1\) and
  \(d_{0i} = 0\):
\item
  \textbf{Always-takers}. The subpopulation with \(d_{1i} =d_{0i} = 1\):
\item
  \textbf{Never-takers}. The subpopulation with \(d_{1i} =d_{0i} = 0\):
\end{itemize}

Using the exclusion restriction, we can define potential outcomes
indexed solely against treatment status using the single-index (
\(\mathrm{Y}_{1 i}, \mathrm{Y}_{0 i}\) ) notation. In particular, \[
\begin{aligned}
\mathrm{Y}_{1 i} & \equiv \mathrm{Y}_{i}(1,1)=\mathrm{Y}_{i}(1,0) \\
\mathrm{Y}_{0 i} & \equiv \mathrm{Y}_{i}(0,1)=\mathrm{Y}_{i}(0,0)
\end{aligned}
\] The observed outcome, \(\mathrm{Y}_{i},\) can therefore be written in
terms of potential outcomes as: \[
\begin{aligned}
\mathrm{Y}_{i} &=\mathrm{Y}_{i}\left(0, \mathrm{Z}_{i}\right)+\left[\mathrm{Y}_{i}\left(1, \mathrm{Z}_{i}\right)-\mathrm{Y}_{i}\left(0, \mathrm{Z}_{i}\right)\right] \mathrm{D}_{i} \\
&=\mathrm{Y}_{0 i}+\left(\mathrm{Y}_{1 i}-\mathrm{Y}_{0 i}\right) \mathrm{D}_{i}
\end{aligned} \] A random-coefficients notation for this is \[
\mathrm{Y}_{i}=\alpha_{0}+\rho_{i} \mathrm{D}_{i}+\eta_{i}
\] with \(\alpha_{0} \equiv E\left[\mathrm{Y}_{0 i}\right]\) and
\(\rho_{i} \equiv \mathrm{Y}_{1 i}-\mathrm{Y}_{0 i}\)

\begin{Shaded}
\begin{Highlighting}[]
\NormalTok{always.takers \textless{}{-}}\StringTok{ }\KeywordTok{with}\NormalTok{(toy\_data, }\KeywordTok{sum}\NormalTok{(D }\OperatorTok{==}\StringTok{ }\DecValTok{1}\NormalTok{)) }\CommentTok{\# n(d,z): n11 + n10 {-}\textgreater{} 30}
\NormalTok{never.takers \textless{}{-}}\StringTok{ }\KeywordTok{with}\NormalTok{(toy\_data, }\KeywordTok{sum}\NormalTok{(D }\OperatorTok{==}\StringTok{ }\DecValTok{0}\NormalTok{)) }\CommentTok{\# n00 + n01  {-}\textgreater{} 40}
\NormalTok{compliers \textless{}{-}}\StringTok{ }\KeywordTok{with}\NormalTok{(toy\_data, }
                  \KeywordTok{sum}\NormalTok{(D }\OperatorTok{==}\StringTok{ }\DecValTok{0} \OperatorTok{\&}\StringTok{ }\NormalTok{Z }\OperatorTok{==}\StringTok{ }\DecValTok{0}\NormalTok{) }\OperatorTok{+}\StringTok{ }
\StringTok{                    }\KeywordTok{sum}\NormalTok{(D }\OperatorTok{==}\StringTok{ }\DecValTok{1} \OperatorTok{\&}\StringTok{ }\NormalTok{Z }\OperatorTok{==}\StringTok{ }\DecValTok{1}\NormalTok{)) }\CommentTok{\# n11 + n00 {-}\textgreater{} 40}
\NormalTok{defiers \textless{}{-}}\StringTok{ }\KeywordTok{with}\NormalTok{(toy\_data, }
                \KeywordTok{sum}\NormalTok{(D }\OperatorTok{==}\StringTok{ }\DecValTok{0} \OperatorTok{\&}\StringTok{ }\NormalTok{Z }\OperatorTok{==}\StringTok{ }\DecValTok{1}\NormalTok{) }\OperatorTok{+}\StringTok{ }
\StringTok{                  }\KeywordTok{sum}\NormalTok{(D }\OperatorTok{==}\StringTok{ }\DecValTok{1} \OperatorTok{\&}\StringTok{ }\NormalTok{Z }\OperatorTok{==}\StringTok{ }\DecValTok{0}\NormalTok{)) }\CommentTok{\# n01 + n10 {-}\textgreater{} 30}

\CommentTok{\# population average treatment effect}
\KeywordTok{with}\NormalTok{(toy\_data, }\KeywordTok{mean}\NormalTok{(y[D }\OperatorTok{==}\StringTok{ }\DecValTok{1}\NormalTok{]) }\OperatorTok{{-}}\StringTok{ }\KeywordTok{mean}\NormalTok{(y[D }\OperatorTok{==}\StringTok{ }\DecValTok{0}\NormalTok{]))}
\end{Highlighting}
\end{Shaded}

\begin{verbatim}
#> [1] 0.45
\end{verbatim}

\begin{Shaded}
\begin{Highlighting}[]
\CommentTok{\# The treatment effect on the treated}
\NormalTok{(ATT \textless{}{-}}\StringTok{ }\KeywordTok{with}\NormalTok{(}\KeywordTok{subset}\NormalTok{(toy\_data, D }\OperatorTok{==}\StringTok{ }\DecValTok{1}\NormalTok{), }
             \KeywordTok{mean}\NormalTok{(y[Z }\OperatorTok{==}\StringTok{ }\DecValTok{1}\NormalTok{]) }\OperatorTok{{-}}\StringTok{ }\KeywordTok{mean}\NormalTok{(y[Z }\OperatorTok{==}\StringTok{ }\DecValTok{0}\NormalTok{])))}
\end{Highlighting}
\end{Shaded}

\begin{verbatim}
#> [1] -0.3
\end{verbatim}

\begin{Shaded}
\begin{Highlighting}[]
\CommentTok{\# TE on compliers}
\KeywordTok{with}\NormalTok{(}\KeywordTok{subset}\NormalTok{(toy\_data, (D }\OperatorTok{==}\StringTok{ }\DecValTok{1} \OperatorTok{\&}\StringTok{ }\NormalTok{Z }\OperatorTok{==}\StringTok{ }\DecValTok{1}\NormalTok{) }\OperatorTok{|}\StringTok{ }\NormalTok{(D }\OperatorTok{==}\StringTok{ }\DecValTok{0} \OperatorTok{\&}\StringTok{ }\NormalTok{Z }\OperatorTok{==}\StringTok{ }\DecValTok{0}\NormalTok{)), }
     \KeywordTok{mean}\NormalTok{(y[Z }\OperatorTok{==}\StringTok{ }\DecValTok{1}\NormalTok{]) }\OperatorTok{{-}}\StringTok{ }\KeywordTok{mean}\NormalTok{(y[Z }\OperatorTok{==}\StringTok{ }\DecValTok{0}\NormalTok{]))}
\end{Highlighting}
\end{Shaded}

\begin{verbatim}
#> [1] 0.2
\end{verbatim}

\begin{Shaded}
\begin{Highlighting}[]
\CommentTok{\# TE on always{-}takers}
\KeywordTok{with}\NormalTok{(toy\_data, }\KeywordTok{mean}\NormalTok{(y[(D }\OperatorTok{==}\StringTok{ }\DecValTok{1} \OperatorTok{\&}\StringTok{ }\NormalTok{Z }\OperatorTok{==}\StringTok{ }\DecValTok{1}\NormalTok{)]) }\OperatorTok{{-}}\StringTok{ }\KeywordTok{mean}\NormalTok{(y[(D }\OperatorTok{==}\StringTok{ }\DecValTok{1} \OperatorTok{\&}\StringTok{ }\NormalTok{Z }\OperatorTok{==}\StringTok{ }\DecValTok{0}\NormalTok{)]))}
\end{Highlighting}
\end{Shaded}

\begin{verbatim}
#> [1] -0.3
\end{verbatim}

\begin{Shaded}
\begin{Highlighting}[]
\CommentTok{\# proportions of subpopulation }
\NormalTok{props \textless{}{-}}\StringTok{ }\KeywordTok{with}\NormalTok{(toy\_data, }
              \KeywordTok{c}\NormalTok{(}
    \KeywordTok{sum}\NormalTok{(Z[D }\OperatorTok{==}\StringTok{ }\DecValTok{1}\NormalTok{] }\OperatorTok{==}\StringTok{ }\DecValTok{0}\NormalTok{) }\OperatorTok{/}\StringTok{ }\KeywordTok{length}\NormalTok{(Z[D }\OperatorTok{==}\StringTok{ }\DecValTok{1}\NormalTok{]), }\CommentTok{\# D\_0i = 1 | D\_i = 1}
    \KeywordTok{sum}\NormalTok{(Z[D }\OperatorTok{==}\StringTok{ }\DecValTok{1}\NormalTok{] }\OperatorTok{==}\StringTok{ }\DecValTok{1}\NormalTok{) }\OperatorTok{/}\StringTok{ }\KeywordTok{length}\NormalTok{(Z[D }\OperatorTok{==}\StringTok{ }\DecValTok{1}\NormalTok{]), }\CommentTok{\# D\_1i = 1 | D\_i = 1}
    \KeywordTok{sum}\NormalTok{(Z[D }\OperatorTok{==}\StringTok{ }\DecValTok{0}\NormalTok{] }\OperatorTok{==}\StringTok{ }\DecValTok{0}\NormalTok{) }\OperatorTok{/}\StringTok{ }\KeywordTok{length}\NormalTok{(Z[D }\OperatorTok{==}\StringTok{ }\DecValTok{0}\NormalTok{]), }\CommentTok{\# D\_0i = 0 | D\_i = 0}
    \KeywordTok{sum}\NormalTok{(Z[D }\OperatorTok{==}\StringTok{ }\DecValTok{0}\NormalTok{] }\OperatorTok{==}\StringTok{ }\DecValTok{1}\NormalTok{) }\OperatorTok{/}\StringTok{ }\KeywordTok{length}\NormalTok{(Z[D }\OperatorTok{==}\StringTok{ }\DecValTok{0}\NormalTok{])  }\CommentTok{\# D\_1i = 0 | D\_i = 0}
\NormalTok{  ))}
\NormalTok{props}
\end{Highlighting}
\end{Shaded}

\begin{verbatim}
#> [1] 0.66667 0.33333 0.75000 0.25000
\end{verbatim}

\end{document}
